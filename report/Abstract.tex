In order to solve the overpopulation issue, scientists are studying Mars features to make it habitable. In this context, a system capable of stabilizing, in real time, the rover arm in front of a Martian rock to allow researchers to study it could be helpful. This project intends to design such a system, through the realization of a depth mapping of the rock thanks to the structured light technique. First, a scene analysis is carried out to determine the camera and the needed aperture to get enough light and to choose an artificial light source capable of outshining the sunlight. To detect the beam spots projected as a unicolor grid by the green LED selected as additional light source, a color detection algorithm is performed in C/C++ with OpenCV, as well as a centroiding algorithm which takes into account the intensity of the pixels belonging to the points. Then, based on the triangulation principle, the distance camera-target is derived and implemented.

The system is verified, with an alternative equipment due to lack of time, through two indoor experiments with one laser beam point and a line of dots. The results are quite promising, since the errors between the calculated distance and the real one could be reduced with an improvement of the calibration and of the color detection. The real time computation is validated. An outdoor experiment has also confirmed that the artificial light source should be capable of outshining the sun on the Red Planet. Nevertheless, extra work is required to implement the real system and the use of patterns, other than a unicolor grid which cannot prevent patterns disorders, could permit to get a more accurate 3D map.