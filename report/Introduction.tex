The research of habitable planet is one of the major concerns of human beings. Indeed, this discovery, in addition to give hope of locating another species, would bring solutions to overpopulation. According to scientists, the main criterion for life is water. Since the properties of celestial bodies are disparate, a habitable zone was defined, considering that water can only exist at a specific range of temperature. If the temperature of the planet is too low, the water will freeze and conversely, it will evaporate if the temperature is too warm. The habitable zone is not fixed; it moves with the evolution of the sun. Mars is thought to have belonged to the zone once, due to its most hospitable climate after Earth. For a matter of fact, NASA's Mars Reconnaissance Orbiter has recently provided a strong evidence of water currently flowing on the Red Planet. It may then be possible to live on Mars and that is the aim of Mars one project: send a colony over there. The terraforming, which would change characteristics of Mars to make it suitable for humans, is also worth considering before colonization. In order to do so, scientists need to learn more about the features of the planet. Orbiters and rovers are already scanning mars surface and soil. Nevertheless, they lack the depth information, only equipped of Two Dimensional (2D) cameras. Their three dimensional (3D) scans have permitted to acquire it and to print a Martian meteorite on Earth in 2014. This information is useful for a better understanding of the rocks properties but that is not enough. Some missing data have delayed the 3D print. It could be interesting to provide to scientists a real time 3D map of a selected rock. It would be easier to study stones but the map could also help the robot to stabilize. With wind and an uneven ground, the rover is liable to move. The pattern recognition of a rock would be more robust by adding a 3D map. Samples could be collected being sure that the robot has taken the right piece. 

Furthermore, the 3D map could have several other applications. The features of the Martian soil acquired by the orbiting satellites are not exhaustive and lack of accuracy. Thus, applied to the area in front of the robot, the 3D map could detect new obstacles and update in real time the available map to distinguish modifications of the ground. 

In this context, designing a system capable of carrying out a 3D map of a Martian rock would be profitable for future research. This is the purpose of this project, undertaken during the Image Analysis with Microcomputer course, under the supervision of Alessandro Massaro. Combining image and scene analysis, programming and robotics seems relevant for learning to work as an engineer.

This report is intended to our fellow students. The first part will bring additional knowledge required to understand the next developments. The design of the system is described in the next section, followed by its verification. To end, future work is addressed.