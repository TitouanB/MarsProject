\section*{Future Work}
\addcontentsline{toc}{part}{Future Work}

Even if the theory section and the implementation of algorithms give a solid basis to achieve the goal of the system and the experiments give good prospects, there is still work to do.

The first step would be carrying out the different experiments once again but with adapted tools in order to obtain a precise calibration. Moreover, the color detection algorithm could be improved in order to detect the same pixels over time. Once these two improvement done, it would be interesting to compute the new errors of the experiment to check weather or not they decrease a lot as expected and if these errors are small enough to achieve a precise determination of the distance board-target. It would also be great to carry out integration tests on the artificial light source and the camera. Indeed, even if the Signal/Noise ratio calculated validates its design, tests have to be performed. As instance it could permit to verify the depth of field, the SNR and if the camera works well with a target between one and two meters. Then, as the algorithms should work with a grid of lighting dots instead of a line of dots but have not been tested, an experiment should be performed. Finally, an integration test should be carried out on the whole system, that is to say the camera plus the artificial light source plus the algorithms.

Moreover, the use of patterns could be 